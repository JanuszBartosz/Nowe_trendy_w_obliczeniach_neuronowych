\documentclass{article}

% if you need to pass options to natbib, use, e.g.:
% \PassOptionsToPackage{numbers, compress}{natbib}
% before loading nips_2017
%
% to avoid loading the natbib package, add option nonatbib:
% \usepackage[nonatbib]{nips_2017}

\usepackage{nips_2017}

% to compile a camera-ready version, add the [final] option, e.g.:
% \usepackage[final]{nips_2017}

\usepackage{polski}
\usepackage[utf8]{inputenc} % allow utf-8 input
\usepackage[T1]{fontenc}    % use 8-bit T1 fonts
\usepackage{hyperref}       % hyperlinks
\usepackage{url}            % simple URL typesetting
\usepackage{booktabs}       % professional-quality tables
\usepackage{amsfonts}       % blackboard math symbols
\usepackage{nicefrac}       % compact symbols for 1/2, etc.
\usepackage{microtype}      % microtypography

\title{Rozpoznawanie stylu budowli na podstawie jej zdjęcia
\\\normalsize Nowe trendy w obliczeniach neuronowych - Projekt }

% The \author macro works with any number of authors. There are two
% commands used to separate the names and addresses of multiple
% authors: \And and \AND.
%
% Using \And between authors leaves it to LaTeX to determine where to
% break the lines. Using \AND forces a line break at that point. So,
% if LaTeX puts 3 of 4 authors names on the first line, and the last
% on the second line, try using \AND instead of \And before the third
% author name.

\author{Bartosz Janusz 210004}

\begin{document}
% \nipsfinalcopy is no longer used

\maketitle

\begin{abstract}
  Celem projektu jest stworzenie klasyfikatora w oparciu o głębokie sieci neuronowe, który dla danych wejściowych w postaci zdjęcia budynku, będzie w stanie przyporządkować go do jednej z kilku epok, której styl reprezentuje. Autor zastrzega sobie możliwość arbitralnego wyboru, które epoki będą obsługiwane (wynikać to będzie z dostępności danych uczących). W ramach projektu zakłada się wykorzystanie istniejącego zbioru anotowanych przykładów oraz wzbogacenie go. 
\end{abstract}

\section{Pierwszy kamień milowy}
Pierwszy kamień milowy:
Zebranie zbioru uczącego – rozszerzenie istniejącego zbioru za pomocą grafik z sieci, wykorzystanie technik data augmentation do jego powiększenia. Aplikacja umożliwiająca preprocessing zdjęcia do wektora liczb gotowego do podania na wejścia sieci neuronowej. Skompletowanie środowiska do przeprowadzenia badań i możliwość uruchomienia uczenia na podstawowym modelu.
\section{Drugi kamień milowy}
Dostrojenie hiperparametrów modelu, działąjący klasyfikator.


\end{document}

Poni¿ej znajduje siê tabela, w której znajduj¹ siê informacje jak zapisaæ poprawnie polskie znaki w formie zakodowanej:
POLSKA LITERA   KOD TEX POLSKA LITERA   KOD TEX
¹       \k{a}   ¥       \k{A}
æ       \'c     Æ       \'C
ê       \k{e}   Ê       \k{E}
³       \l{}    £       \L{}
ñ       \'n     Ñ       \'N
ó       \'o     Ó       \'O
œ       \'s     Œ       \'S
¿       \.z     ¯       \.Z
Ÿ       \'z            \'Z

